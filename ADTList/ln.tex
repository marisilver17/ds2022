%!Tex Program = xelatex
%\documentclass[a4paper]{article}
\documentclass[a4paper]{ctexart}
\usepackage{xltxtra}
%\setmainfont[Mapping=tex-text]{AR PL UMing CN:style=Light}
%\setmainfont[Mapping=tex-text]{AR PL UKai CN:style=Book}
%\setmainfont[Mapping=tex-text]{WenQuanYi Zen Hei:style=Regular}
%\setmainfont[Mapping=tex-text]{WenQuanYi Zen Hei Sharp:style=Regular}
%\setmainfont[Mapping=tex-text]{AR PL KaitiM GB:style=Regular} 
%\setmainfont[Mapping=tex-text]{AR PL SungtiL GB:style=Regular} 
%\setmainfont[Mapping=tex-text]{WenQuanYi Zen Hei Mono:style=Regular} 

\usepackage{listings}
\usepackage{xcolor}

\definecolor{codegreen}{rgb}{0,0.6,0}
\definecolor{codegray}{rgb}{0.5,0.5,0.5}
\definecolor{codepurple}{rgb}{0.58,0,0.82}
\definecolor{backcolour}{rgb}{0.95,0.95,0.92}

\lstdefinestyle{mystyle}{
    backgroundcolor=\color{backcolour},   
    commentstyle=\color{codegreen},
    keywordstyle=\color{magenta},
    numberstyle=\tiny\color{codegray},
    stringstyle=\color{codepurple},
    basicstyle=\ttfamily\footnotesize,
    breakatwhitespace=false,         
    breaklines=true,                 
    captionpos=b,                    
    keepspaces=true,                 
    numbers=left,                    
    numbersep=5pt,                  
    showspaces=false,                
    showstringspaces=false,
    showtabs=false,                  
    tabsize=2
}

\lstset{style=mystyle}

\title{表, 堆栈和队列}
\author{王何宇}
\date{}
\begin{document}
\maketitle
\pagestyle{empty}

本章讨论最简单和基础的数据结构(data structure). 我们会介绍抽象数据类型
(Abstract Data Type, ADT) 的概念;与之相关的操作;以及 C++ 的实现.
本课程不会专门留出时间介绍 C++ 编程, 而是在实际学习过程中不断引入 C++
的概念和实现. 我们需要具备在一个实际工作中不断学习的能力. 当然, 如果你具备一定的
C++ 及编程的基础, 无疑会提高学习和工作的效率. 但我们并不假定这一条件总是具备.

可以认为 ADT 是一个抽象的数学概念. 我们定义它为一个由基本元素构成的集合,
以及相应的操作(operation). 比如我们可以定义一个数据类型名为 set,
一个 set 可以包含有限个或零个元素. 我们可以定义操作为 add, 即增加一个指定的元素;
操作 remove, 删除一个指定的元素;操作 size, 计算当前元素总数, 等等.

不同的编程语言, 都不难将上述概念具体化. 比如将元素具体为一个浮点数,
然后我们可以讨论如何实现一个有限的浮点数集, 并将其用于我们具体的问题中. 而 C++
提供了面向对象设计(object-oriented)的能力, 因此可以将这件事情做的更加直接. 一个
C++ 类(class), 允许将数据结构和操作捆绑在一起定义, 因此几乎可以直接视为一个 ADT
的实现.

\section{抽象数据结构: 表(List)}

我们称有限个元素组成的序列 $A_0$, $A_1$, $\cdots$, $A_{N - 1}$ 为一个 ADT List.
表的长度 (size) 为非负整数 $N$. 如果 $N == 0$ 为真, 我们称该表为空表(empty list).

对非空表中元素 $A_i$, $0 \leq i < N$, $A_{i - 1}$ 称为它的前任 (predecessor),
$A_{i + 1}$ 称为它的继任 (successor). 当然, 前任和继任可以不存在. 而自然数 $i$ 称为
$A_i$ 的位置 (position).

我们可以定义以下一些常见的操作:
\begin{itemize}
\item
\begin{verbatim}
printList
\end{verbatim}
输出表中全部元素. 空表则什么也不输出.

\item
\begin{verbatim}
makeEmpty
\end{verbatim}
清空列表, 删除全部元素. 空表则什么也不发生. 

\item
\begin{verbatim}
find (x)
\end{verbatim}
这里 \verb|x| 是一个具体的值, 表中寻找值 \verb|x|, 如果找到, 返回第一
个找到的 \verb|x| 的位置, 否则返回 \verb|nil|. 这里我们用 \verb|nil|
代表一个否定的结果, 在具体实现时,可以用一个指定的数或指针.

\item 
\begin{verbatim}
insert (x, pos)
\end{verbatim}
这里 \verb|x| 是一个具体的值, \verb|pos| 是一个位置, 即在指定位置
\verb|pos| 前插入一个新元素 \verb|x|, 插入之后新元素 \verb|x| 在表中的
位置将会是 \verb|pos|. 如果 \verb|pos| 位置不存在, 则报错.

\item
\begin{verbatim}
remove (x)
\end{verbatim}
删除 \verb|x|, 如果表中没有 \verb|x| 则什么事也不发生. 这里我们可以规
定, 如果表中不止一个 \verb|x|, 一次只删除第一个发现的那一个. 也即这里
其实是先操作了一次 \verb|find|. 

\item
\begin{verbatim}
findKth (pos)
\end{verbatim}
返回第 $pos$ 位置的元素. 如果位置 \verb|pos| 不存在, 则报错.

\item
\begin{verbatim}
next (pos)
\end{verbatim}
返回第 $pos$ 位置的继任元素的位置. 如果继任不存在, 则返回 NIL.

\item
\begin{verbatim}
previous (pos)
\end{verbatim}
返回第 $pos$ 位置的前任元素的位置. 如果前任不存在, 则返回 NIL.
\end{itemize}

我们还可以根据需要, 定义和扩展表的操作功能. 而对于不同的具体实现, 这些
功能定义也可能得不到严格的遵守(因此需要作出调整和改变).

在以上操作中, 我们发现有些操作是不会改变表的内容和结构的, 如
\verb|printList| 和 \verb|find|, 这些操作我们一般称为静态操作;而有些
操作则会改变, 如 \verb|insert| 和 \verb|remove|. 这里我们称其为动态操
作.

\subsection{表的简单数组(Array)实现}
数组(Array)是指用一段{\bf 连续}的内存空间来存放一个表. 由于表的每一个
元素所占内存空间相等, 因此我们只需记录表的第一个元素所在内存地址, 那么
就可以容易地获得第 $k$ 个元素的内存地址. 在 C / C++ 中, 这个 findKth
的操作, 直接通过运算 \verb|[]| 实现. 这种实现不但非常简单, 而且是计算
机存储在系统层面上最基础的方式, 本质上, 我们的全部内存都可以看作按内存
地址连续存放的一个大数组.

得益于数据在内存中的连续存放, 静态操作如 \verb|findKth| 的时间代价一般
为 $\O(1)$, 而动态操作如 \verb|insert| 和 \verb|remove| 的代价是
$\O(N)$. 这是因为我们在改变了表的结构之后, 必须确保数据仍然有序和连续存放. 

换言之, 用 Array 去实现 ADT List, 静态效率较高而动态效率较低. 

我们接下去用 C++ 实现这个操作, 主要目的是用这个简单的模型, 学习 C++ 的
基本框架.

C++ 的核心模型是类(Class). 以类为核心的编程模型又被成为面向对象编程.
它要求我们在建模时, 一切围绕类展开. 而一个类, 本质上就是对一个基本数据
结构的描述, 或者说, 就是 ADT 的一个具体实现. 它要求我们定义两个核心部分:
\begin{enumerate}
\item 有哪些数据;
\item 有哪些操作.
\end{enumerate}
注意这个定义和 ADT 一样是抽象的. 比如我们定义学生这个概念类:
\begin{verbatim}
class 学生
{
/// 数据: 
    int 学号;
    bool 性别;
    double 体重;
/// 操作:
    吃(double x); // 体重增加 x.
    跑(double t); // 体重减轻 t * 0.1. 
    变性();  // 如果成功, 性别取反.
    ...
}
\end{verbatim}

注意这个概念是抽象的, 它定义了一个作为概念的学生. 根据这个概念, 我们可
以创建一个具体实体, 称为一个实例. 比如: 张三是一个学生.
\begin{verbatim}
学生 张三;
\end{verbatim}
那么在这个模型下, 我们就可以对张三作出一些操作和改变:
\begin{verbatim}
学生 张三;
张三.吃(10);
张三.跑(30);
张三.变性();
\end{verbatim}

C++作为一门面向底层建模的语言, 是非常强调安全性和严格的. 因此 C++ 对数
据的可操作性也做了严格的规定. 具体说就是提供了关键字:
\begin{itemize}
\item \verb|private|, 私有的, 只能被类内部的成员函数调用和访问;
\item \verb|public|, 共有的, 可以被类内和外部的函数调用和访问.  
\end{itemize}

其实还有一个关键字 \verb|protected| 与此有关, 但我们稍后再讲. 此外, 作
为例外处理考虑, 如果有一个外部的类或者函数想要访问一个类的
\verb|private| 成员, 则可以通过关键字 \verb|friend| 声明例外. 我们现在
就只要有这些印象, 具体的使用规范我们到实例中讲.

比如, 上面的学生类, 也许更严格的规范是这样的:
\begin{verbatim}
class 学生
{
/// 数据:
private: 
    int 学号;
    bool 性别;
    double 体重;
/// 操作:
public:
    吃(double x); // 体重增加 x.
    跑(double t); // 体重减轻 t * 0.1. 
    变性();  // 如果成功, 性别取反.
    ...
}
\end{verbatim}

从一个类产生一个具体的实例, 有一些特殊的时间节点, 需要我们格外关注:
\begin{itemize}
\item 当一个实例被创建和初始化数据时;
\item 当一个实例被删除时;
\item 从一个已存在的实例复制一个新的实例;
\end{itemize}
C++ 为这些时刻做了专门的规定, 也就是我们课本 1.5.6 描述的 The Big
Five.

首先介绍两个最基本的规定:
\begin{itemize}
  \item 构造函数, constructor. 规定必须有一个函数, 函数名和类名一致,
    没有返回类型. 参数可以自定. 这是在创建实例时, 会第一个调用的函数,
    因此可以在这里规定实例的数据如何初始化.
  \item 析构函数, destructor. 规定必须有一个函数, 函数名为\~类名, 同样
    没有返回类型, 一般也没有参数. 这是当一个实例正常删除时, 最后被调用
    的一个函数, 在这里可以做一些收尾工作, 比如确保动态分配的内存都已经
    安全释放.
\end{itemize}
如果没有显式声明构造和析构函数, 那么编译器也会默认产生一个缺省的构造函
数和析构函数. 它们的都相当于一个带空参数的空函数. 和 C++ 其他函数一样,
构造函数可以重载, 但析构不会重载(为什么?).

现在来考虑如何设计和实现 ArrayList. 这里我们不实现 \verb|next| 和
\verb|previous| 操作, 因为过于平凡. 而且一旦增加这两个操作, 我们就必须
在数据成员中增加一个 currentPosition. (为什么?)

以下是类的设计:

\begin{lstlisting}[language=C++]
#ifndef __CRAZYFISH_ARRAYLIST__
#define __CRAZYFISH_ARRAYLIST__

#include <iostream>

/// 元素的类型.
typedef double t_ele;

/// 元素下标的类型.
typedef int t_idx;

/// 代表不存在的下标位置.
#define NIL -1

class ArrayList
{
private:
	t_idx size = 0;			/**< List 的大小. */
	t_ele *data = nullptr;		/**< 存放数据的内存区域. */

public:
	/** 
	 * 默认构造函数.
	 * 
	 */
	ArrayList(){};

	/** 
	 * 析构函数.
	 * 
	 */
	~ArrayList();

	/** 
	 * 列出表内全部元素.
	 * 
	 */
	void printList() const;

	/** 
	 * 删除表内全部元素, 使之成为空表.
	 * 
	 */
	void makeEmpty();

	/** 
	 * 查找指定元素. 
	 * 
	 * @param _val 被查找的元素.  
	 * 
	 * @return 第一个找到的元素的位置, 如果没有找到, 返回 NIL.
	 */
	t_idx find(t_ele _val) const;
	
	/** 
	 * 在指定位置插入指定元素值. 原位置元素和一切后续元素依次后移.
	 * 
	 * @param _val 插入的元素.
	 * @param _pos 指定位置. 如果位置错误则报错. 
	 */
	void insert(t_ele _val, t_idx _pos);

	/** 
	 * 删除第一个指定元素值(如有), 一切后续元素(如有)依次前移. 
	 * 如果不存在该值元素, 则什么也不发生.
	 * 
	 * @param _val 指定元素值.
	 */
	void remove(t_ele _val);

	/** 
	 * 返回 List 的指定位置元素值.
	 * 
	 * @param _pos 指定位置.
	 * 
	 * @return 指定元素值.
	 */
	t_ele findKth(t_idx _pos) const;
};
#else
/// DO NOTHING.
#endif
\end{lstlisting}

这里说明几点:

\begin{itemize}
  \item 关键字 \verb|const| 放在函数后面, 表示这个函数不会改动类的数据
    和结构, 可以认为是一个静态操作的标记;
  \item 由于还未学习 C++ 的很多特性, 因此这里采用了宏和 \verb|typedef|
    对数据做了抽象. 在后续的学习中, C++ 会有更好的机制;
  \item 这里的注释符合 Doxygen 的规范, 便于自动生成文档. 这也是推荐的
    注释格式. 建议使用适当的工具辅助你产生规范的注释;
  \item 除了 \verb|next| 和 \verb|previous|, 这个设计也没有考虑出错和
    报错. 留待后续讨论.
  \item 关键字 \verb|nullptr| 是 C++ 0x 标准中新增的, 用来代替宏
    \verb|NULL|.
\end{itemize}

我们在测试时注意到, 已现有的框架, 要初始化一个列别是繁琐的. 比如为了实
现课本 3.2 导言中的例子, 我们不得不执行这段代码:

\begin{verbatim}
ArrayList A;
A.insert(34, 0);
A.insert(12, 1);
A.insert(52, 2);
A.insert(16, 3);
A.insert(12, 4);
\end{verbatim}

为此, 我们重载一个新的构造函数:

\begin{verbatim}
/** 
* 初始化构造函数.
* 
* @param _l 初始化列表.
*/
ArrayList(std::initializer_list<t_ele> _l);
\end{verbatim}
这里用到了 C++11 的新特性 \verb|initializer_list|. 我们需要引用头文件:
\begin{verbatim}
#include <initializer_list>
\end{verbatim}
我们在实现它时, 也用了一些 C++11 的新特性:
\begin{verbatim}
ArrayList::ArrayList(std::initializer_list<t_ele> _l)
{
	size = _l.size();
	data = new t_ele[size];
	t_idx i = 0;
	for (t_ele e : _l)
		data[i++] = e;
};
\end{verbatim}
注意这个 \verb|for| 的新颖的格式, 具有浓郁的 Python 风格了. 它的语法意
义是``\verb|e| 依次取遍 \verb|_l| 中的每一个 \verb|t_ele| 值''.

有了这个初始函数, 我们的测试初始化就简单了:
\begin{verbatim}
ArrayList A{34, 12, 52, 16, 12};
\end{verbatim}

在测试 \verb|insert| 和 \verb|findKth| 时, 如果用户提供了错误的
\verb|_pos|, 那么会导致数组越界, 但编译器语法检查会无视这一点. 甚至在
测试运行时, 这样的语句也未必会导致系统运行错误.
\begin{verbatim}
std::cout << A.findKth(10) << std::endl;
\end{verbatim}
我们只有在做内存泄漏检查时才能发现这一点:
\begin{verbatim}
valgrind ./test
\end{verbatim}
为了解决这个问题, 我们需要自己制定错误异常检测和退出机制. 我们现在并不
打算介绍完整的 C++ 异常信号机制, 而是用手工代码来强调这个问题的必要性.
我们需要在每次外部调用位置信息的时候, 对它是否越界做一个检查: 
\begin{verbatim}
t_ele ArrayList::findKth(t_idx _pos) 
{
     if (_pos > size)
         _error_exit(1);
     return data[_pos];
};
\end{verbatim}
然后设计一个类的内部(私有)函数来统一处理出错信息. 因为我们可能的错误也
许不止这一个.
\begin{verbatim}
/** 
* 异常退出.
* 
* @param _err_code 出错代码.
*  1 : 越界错误.
* 
*/
void _error_exit(t_err _err_code);
\end{verbatim}
由此, 我们还需要规定一个统一的出错码 \verb|_err_code|, 以及它的类型
\verb|t_err|. 作为退出函数, 我们则调用了 \verb|cstdlib| 库中的
\verb|exit| 函数, 并用一个 $-1$ 参数表示这不是一个正常的退出. 这个信息
会通过一种信号机制传递给操作系统. 我们这里先不用去理会.

现在, 我们的实现会捕捉越界错误, 并且给出警告后强制退出. 作为引入异常处
理机制的最后一个需要提醒的地方, 我们注意到 \verb|findKth| 函数, 不能再
在后面施加 \verb|const| 参数了. 尽管它概念上仍然是一个静态操作, 但它有
可能因为下标越界而强制退出, 这是一个非静态的行为. 
\begin{verbatim}
void ArrayList::_error_exit(t_err _err_code)
{
  switch (_err_code)
  {
    case 1:
      std::cerr << "Out of the range of the position." << std::endl;
    default:
      makeEmpty();
      std::exit(-1);
  }
};
\end{verbatim}

\subsection{表的单链表(SingleLinkedList)实现}
在这种实现中, 我们将基本数据单元组织成一个节点(node), 每一个 node 由数
据和下一个节点的地址构成. 这样形成的表结构称为单链表(Single Linked
List). 当然也可以在节点中增加上一个节点的地址, 从而构成双链表(Double
Linked List). 我们这里演示一个单链表的实现. 首先来设计其节点.

\begin{verbatim}
class Node
{
private:
  t_ele data;
  Node* next = nullptr; 
  Node(){};
  Node(t_ele _val){data = _val;};
  ~Node(){};
};
\end{verbatim}
这里我们对 \verb|Node| 的定义中用到了自身 \verb|Node*|, 这种递归的定义
在 C 语言年代就是允许的.

我们这里沿用了上一个实现中对 \verb|t_ele| 的定义. 这种方式有一个严重的
问题就是一旦用户想要修改 \verb|t_ele|, 比如改成一个结构体, 那么他必须
修改源代码, 然后重新编译全部文件. 这对于代码稳定来说是危险且低效的. 也
不符合 C++ 追求极致抽象的精神. 因此 C++ 提供了模板机制(template), 在函
数库中只规定抽象数据类型和具体数据类型的对应规则, 具体的指定交给用户去
决定. 但这也意味着, 类实际上是在预编译阶段才真正确定的. 这就导致较难像
之前那样, 分离 .h 头文件和函数实现的 .cpp 文件, 也较难通过提供 .h 和
.o 文件给用户使用. 当然也许根本没这个必要, 如果你是开源的信徒, 那么直
接把全部内容都写在 .h 文件中就可以了. 目前, 代表 C++ 语言未来标准的
Boost 库就是这样做的.

我们现在来改写我们的节点设计:
\begin{verbatim}
template <typename T>
class Node
{
private:
  T data;
  Node* next = nullptr; 
  Node(){};
  Node(T _val){data = _val;};
  ~Node(){};
};
\end{verbatim}
现在原本的 \verb|t_ele| 被我们用数据类型模板参数 \verb|T| 代替. 它和
\verb|typedef| 最大的区别是, 这个 \verb|T| 具体是什么数据类型, 是由使
用这个库的用户在写应用代码时决定的. 比如用户想用 \verb|double| 作为实
际类型, 那么他就可以这样声明一个节点:
\begin{verbatim}
Node<double> A;
\end{verbatim}
当用户的代码被编译时, 编译器会在预处理阶段, 直接根据模板产生真正包含实
际类的代码, 然后再编译. 我们接下去都会使用模板.

接下去一个问题是, 我们接下去的设计, 会出现两个类, 节点和链表. 它们之间
的关系是什么? 对象之间的关系, 有很多种, 常见的有:
\begin{itemize}
  \item A is a B, 也即 A 是具体的, 而 B 是抽象的. 如张三是一个人. 然后我们设
  计时, 将人设计成一个类, 然后张三是人的一个实例. 这种就是典型的 is a 关系.
  \item A has a B, 也即 B 是 A 的一部分. A 和 B 可以都是抽象的. 比如人
    有一个头. 这种抽象层面的定义会直接影响到全部具体层面的实例. 现在张
    三作为一个人, 必须有一个具体的头. 
\end{itemize}
现在我们注意到, 链表 has a 节点. 所以, 节点应该作为链表的一个成员是无
异议的. 但节点这个概念, 应该独立于链表存在么? 比如头这个概念, 可以独立
于人存在. 猪也可以有猪头. 但如果头和人这两个概念完全各自独立, 又各自可
以实例化, 那么就会出现: 张三是一个人, 猪头是一个头, 张三可以长一个猪头.
所以这种抽象的讨论是有意义的, 它规定了当我们的概念在具体化时, 能够表现
的范畴. 这里作为设计师要考虑的问题就是, 是否应该或有必要让这种异化的实
例存在. 比如在我们的设计中, 如果我们的节点的类定义也放到单链表定义中去,
那么就意味着我们给单链表单独定制了一种节点, 该节点除了单链表, 不会(不
允许)有任何其他应用场合. 如果你说, 从代码复用的角度出发, 也许节点这种
结构, 可能有更多的数据结构应用场合, 因此不如将它定义在链表类之外, 然后
根据需要拼装. 我们甚至可以定义单链表节点和双链表节点, 然后直接拼装成单
链表和双链表. 但这么做的问题是, 你也可能错误地将双链表节点用于单链表.
而且这种错误, 在使用中不会有任何逻辑错误, 但会不必要占用额外的资源, 并
增加内存泄漏的风险.

这一段讨论, 提示我们封装和代码复用都是 C++ 的基本原则, 但它们也可能需
要做一个平衡. 在这里, 我个人坚持代码的安全和稳定高于复用. 同时, 为了简
化实现, 我将把单链表节点的定义直接放到单链表的定义内部去. 这种实现, 等
于说, 节点是一个依附于链表的概念. 该头之所以是人头, 因为它定义在人内部,
所以任何人的实例上不可能存在猪头, 哪怕它长得像猪头, 它按定义也是人头.
未来等我们设计水平提高了, 也许会有更好的策略.

\begin{itemize}
  \item {\bf 封装原则}: 每一个属性和函数, 在安全和必要的前提下, 都要严
    格约束它的作用范围和可视范围. 在抽象概念内部的属性, 应该通过接口函
    数和外部交换数据. (只有张三能动自己的头, 理论上...)
  \item {\bf 代码复用}: 在一个独立代码系统中, 尽量避免出现内容或效果完
    全一致的两段代码. 应该通过函数等统一调用. 这不仅仅是为了节省代码量,
    更重要的是避免冗余操作和数据. (如果要砸头, 不管是张三的头还是猪头,
    同一根棍子砸最公平...)
\end{itemize}

现在看一下我们的单链表设计:

\begin{lstlisting}[language=C++]
#ifndef __CRAZYFISH_SINGLELINKED_LIST__
#define __CRAZYFISH_SINGLELINKED_LIST__

#include <iostream>
#include <limits>
#include <initializer_list>

/// 元素下标的类型.
typedef int t_idx;

/// 出错代码类型.
typedef int t_err;

template <typename T>
class SingleLinkedList
{
private:
    class Node
    {
        T data;			/**< 节点内存放的数据. */
        Node* next = nullptr;	/**< 指向下一个节点的指针. */

        /** 
         * 带数据初始化的构造函数.
         * 
         * @param _val 初始化的数据.
         */
         Node(const T& _val){data = _val;};

         friend class SingleLinkedList<T>;
    };
    Node* head = nullptr;	/**< 头指针. */
    Node* currentPos = nullptr;	/**< 当前位置的指针. 除非是空链表, 否则该指针不能空. */
    t_idx size = 0;		/**< 节点总数. */

    /** 
     * 异常退出.
     * 
     * @param _err_code 出错代码.
     *  1 : 越界错误.
     * 
     */
    void _error_exit(t_err _err_code);

public:
    /** 
     * 默认构造函数.
     * 
     */
    SingleLinkedList(){};

    /** 
     * 默认析构函数.
     * 
     */
    ~SingleLinkedList();

    /** 
     * 带列表初始化的构造函数.
     * 
     * @param _l 初始化列表.
     */
     SingleLinkedList(std::initializer_list<T> _l);

    /** 
     * 列出表内全部元素.
     * 
     */
     void printList() const;

    /** 
     * 删除表内全部元素, 使之成为空表.
     * 
     */
     void makeEmpty();

    /** 
     * 查找指定元素.
     * 
     * @param _val 被查找的元素.  
     * 
     * @return 将当前位置移动到第一个找到的元素, 如果没有找到, 则当前
     * 位置不动.
     */
     void find(const T& _val);

    /** 
     * 在当前位置之后插入指定元素. 空表则直接插入作为第一个元素. 插入
     * 完成后, 新的当前位置为插入的元素.
     * 
     * @param _val 插入的元素值.
     */
     void insert(const T& _val);

    /** 
     * 如果是空链表, 或者当前位置指向最后一个元素, 则什么也不做;如果
     * 只有一个元素, 删除该元素成为空链表;如果不止一个元素, 且当前位
     * 置不是最后一个元素, 则删除当前位置的下一个元素. 删除之后, 当前
     * 位置不动.
     * 
     */
     void remove();

    /** 
     * 返回指定序号元素值. (低效!) 如果链表内的元素不到 k 个, 则给出警
     * 告.
     * 
     * @param _pos 指定位置.
     * 
     * @return 指定元素值.
     */
     T& findKth(t_idx _pos);

    /** 
     * 将当前位置后移一位, 如果已经是最后一个元素, 则不变.
     * 
     */
     void next();

    /** 
     * 将当前位置前移一位, 如果已经是 head, 则不变. (低效!)
     * 
     */
     void previous();

    /** 
     * 判定链表是否为空.
     * 
     * 
     * @return 空链表返回 true, 非空返回 false.
     */
     bool is_empty() const;

    /** 
     * 得到当前位置的元素值. 静态版本.
     * 
     * 
     * @return 当前位置的元素值的静态引用.
     */
     const T& getCurrentVal() const; 
};
\end{lstlisting}

我们仍然保留了
\begin{verbatim}
/// 元素下标的类型.
typedef int t_idx;

/// 出错代码类型.
typedef int t_err;
\end{verbatim}
因为这两种类型并不会频繁修改, \verb|typedef| 相对更加方便. 而对于数据
本身的类型, 则用模板机制进行了抽象. 这样用户就可以在写应用代码时, 具体
决定用什么类型. 因为链表是一种动态内存管理机制, 因此相对位置比绝对位置
更有意义. 所以我们实现了 \verb|next| 和 \verb|previous| 操作. 为了配合
这两个操作, 我们在基本属性中增加了 \verb|currentPos|. 请思考一下这个属
性的必要性. 这里, 我们规定 \verb|currentPos| 除非在空链表中, 否则必须
指向一个实际的元素, 不能是 \verb|nullptr|. 这个属性的引入, 其实增加了
很多额外的工作量. 比如我们必须始终考虑确保这个属性的值的正确性. 特别是
在 \verb|find| 操作中, 如何合理地规定它的变化. 同时, 这也导致
\verb|find| 不再能保持为一个静态操作. 事实上, 对每一个动态操作, 我们都
要考虑每一个成员属性的正确性和一致性. 比如这里, 逻辑上我们可以通过
\verb|currentPos| 是否为 \verb|nullptr| 来判断这个链表是否为空链表. 但
这个做法并不好. 因为它依赖于我们对 \verb|currentPos| 的特殊规定, 因此
不如判定 \verb|head| 是否为 \verb|nullptr| 自然. 但这就导致一个一致性
问题, 我们必须确保在空链表中, \verb|currentPos| 和 \verb|head| 总都是
\verb|nullptr|. 如果不一致, 那么说明代码出了问题. 合理的做法是在
\verb|is_empty| 中增加判定语句或断言 (assert) 来保证. 这就会更增加我们
的代码量. 因此, 不要增加不必要的属性, 此外, 这里大家可以再体会一下确保
只有一个 \verb|is_empty|, 也就是代码复用的意义所在.

关于模板类的函数实现, 有两个需要注意的问题. 首先, 如果没有必要, 不要再
去考虑代码分离的问题, 也就是把实现单独写成一个 .cpp 文件是繁琐且无聊的.
在开源机制下, 不如全部都写在 .h 文件内. 其次, 因为类的申明带有抽象的模
板参数, 因此对每一个成员函数, 也必须跟上这种模板参数抽象机制. 也就是说,
模板类的成员函数, 天然就是带模板参数的模板函数. 看一个具体的例子:
\begin{verbatim}
template <typename T>
void SingleLinkedList<T>::insert(const T& _val)
{
    ...
};
\end{verbatim}
这里第一行表示下面定义的内容, 带有一个表示类型的模板参数 \verb|T|, 这
里参数标识名一般需要和模板类一致, 否则会有严重的可读性问题. 第二行出现
了两个 \verb|T|, 第一个是一个模板类的具体化, 也就是把第一行的参数
\verb|T| 传递过来, 然后形成一个具体的类 \verb|SingleLinkedList<T>|, 因
此这个参数是不可缺少的, 而且本质上, 我们在一次编译中, 只实现一个针对具
体 \verb|T| 的具体版本 (这里的逻辑关系和规则, 认真考虑一下!);而第二个
\verb|T| 则是对第一行 \verb|T| 的直接引用.

这里顺便说一下, 如果大家觉得每一个函数前都要写一行
\begin{verbatim}
template <typename T>
\end{verbatim}
太烦, 可以考虑用
\begin{verbatim}
#define TEMPLATE template <typename T>
\end{verbatim}
来替换, 特别是当模板参数非常多或复杂时.

对指针操作还不熟悉的同学, 请对照代码和课本仔细推敲. 特别是
\begin{verbatim}
template <typename T>
void SingleLinkedList<T>::insert(const T& _val)
{
	Node *p = new Node(_val);
	if (is_empty())
		head = p;
	else
	{
		p->next = currentPos->next;
		currentPos->next = p;
	}
	currentPos = p;
	size++;
};

template <typename T>
void SingleLinkedList<T>::remove()
{
	if (is_empty() || currentPos->next == nullptr)
		;
	else if (head->next == nullptr)
		makeEmpty();
	else 
	{
		Node* p = currentPos->next;
		currentPos->next = currentPos->next->next;
		delete p;
		size--;
        }
};
\end{verbatim}
这两个函数的实现, 请考虑这里 \verb|Node* p| 的必要性和 \verb|delete|
的执行时机.

此外要注意, \verb|findKth| 和 \verb|previous| 这两个函数都是
$\Theta(n)$ 代价的 (请证明一下). 因此效率低下, 在实际的单链表应用中,
不会出现.

对于动态内存操作, 一个常见的错误是忘记释放内存, 这将导致内存泄漏. 系统
提供了 \verb|valgrind| 工具, 可以做检查. 当然全局调试工具 \verb|gdb|
也可以做到. 

接下去 3.3 \~ 3.5 节, 课本分析了标准 C++ 库 STL 里面 \verb|vector| 和
\verb|list| 的实现, 并给出了完整的代码. 请认真阅读, 学习. 课本的代码是
一个简化版本, 但可以用. 请大家注意:

\begin{itemize}
\item \verb|vector| 中通过重载运算符 \verb|[]| 使之读取接口和标准数组一致;
\item Iterator 方法如何确保容器(container, 包括 \verb|vector| 和
  \verb|list|) 不发生越界错误, 特别是如何通过重载运算符, 保持具有一致
  的操作接口;
\item 继承的运用, 这里普通的 Iterator 是 Const Iterator 的一个派生. 
\end{itemize}

\section{堆栈和队列}

堆栈(stack)和队列(queue)可以认为是表的一个应用. 它们的基本结构和操作并
没有发生实质变化, 它们可以用数组或者链表作为底层实现, 主要特性和区别发
生在如何使用上. 这里我们根据需要和惯例, 给出一些特殊的表操作:
\begin{itemize}
  \item {\bf push, 推入} 向表插入(insert)一个新元素;
  \item {\bf pop, 弹出} 从表内弹出一个元素, 并将它从表中删除. 以上两个
    操作, 因为效率的原因, 我们一般只考虑发生在表的开头或结尾.
  \item {\bf top, 读出} 读取表的一个元素. 它相当于不删除的 pop 操作;
  \item {\bf append, 加入} 向表的尾部插入一个新元素.  
\end{itemize}

以上这些特异化的操作, 是为了堆栈和队列的特殊要求定制的. 堆栈和队列本质上都是表, 主要区别是
每一次 pop 或者 top 得到的元素,
\begin{itemize}
\item {\bf 堆栈} 是最近一次 push 的元素, 先入先出(First in First Out,
  FIFO).
\item {\bf 队列} 是表中最早 push 的元素, 后入先出(Last in First Out,
  LIFO).
\end{itemize}

显然, 队列这种特性是为了对排队建模. 比如我们有一台网络打印机, 然后一个
楼层的同学都在使用. 大家都通过局域网发送打印请求, 打印服务器就会构建一
个队列, 一边按照收到请求的先后次序向一个等待读列推入请求;另一边独立地
把队列中的请求, 按照先后次序弹出并执行. 这种需求在日常工作中非常常见.

堆栈据信是图灵提出的一种机制. 目的是正确处理函数的调用. 比如一个函数
A, 执行的时候调用了函数 B, 函数 B 又调用了函数 C, 函数 C 还调用了 函数
B 和 A, 问正确的函数执行顺序是什么? 最后一次 B 调用结束了应该回到哪个
位置? 在实际工程中, 这种重叠交叉的嵌套调用可以有几千层. 用堆栈的模型很
容易解决这个问题. 我们只要记下调用发生的先后次序:

\begin{center}
A -> B -> C -> B-> A
\end{center}

然后按照堆栈先进先出的原则处理就可以了. 也就是这里真正最先完成的函数是
最后一个 A, 然后返回 B, B 完成了返回 C, 再依次是 B 和 A. 除了函数调用,
在一些括号匹配等问题上, 堆栈也能很方便地解决.

显然, 堆栈和队列概念都是简单的, 我们直接来看一下如何应用于具体问题. 我
们将采用 STL 提供的 stack 和 queue 来完成我们的例子. 顺便熟悉一下 C++
STL.

\subsection{堆栈应用: 四则混合计算器}

用户输入一串字符串, 如果它构成有意义的四则混合运算表达式, 则给出计算结
果, 否则给出错误提示. 例如, 输入
\begin{verbatim}
3.1 * (1.7 + 2.1) / 5.0
\end{verbatim}
输出 \verb|2.356|. 

我们的表达式符合 C / C++ 的规范, 即有意义的字符是 $0 \sim 9$, $+ - *
/$, 空格(包括 TAB, 都忽略), 括号(只允许小括号, 但可以嵌套).

这里我们采取尽可能简单的设计思路. 按以下步骤实施:

\begin{enumerate}
\item 排除掉全部错误的表达式和干扰的符号(如连续多个空格或 TAB);
\item 检查括号的匹配性, 将表达式转换成便于计算的格式;
\item 计算表达式, 给出结果.
\end{enumerate}

第一步, 依次读入字符, 一旦发现非法字符, 停止, 输出错误. 否则过滤掉空格,
形成一个新的干净的表达式. 这里可以用一个字符队列. 当输入过程变得复杂,
比如不止一个输入对象, 而且会发生并发, 锁死时, 采用队列是正确的做法.

接下去, 我们人类之所以要使用括号, 其实是因为我们采用了中缀表达式(infix
expression):
\begin{verbatim}
操作数1 运算符 操作数2 
\end{verbatim}
这样当\verb|操作数2|嵌套了表达式时, 我们就不得不考虑运算优先级的问题,
最终导致了
\begin{itemize}
\item 先乘除, 后加减;
\item 括号优先级最高.
\end{itemize}
这两条奇葩的规定. 这本质上是因为我们有一个线性思考的大脑. 而计算机则表
示, 运算优先级是一个伪命题. 你要做的是, 改变书写习惯, 成为后缀表达式
(postfix expression), 又称逆波兰式(reverse Polish notation):
\begin{verbatim}
操作数1 操作数2 运算符  
\end{verbatim}
比如
\begin{verbatim}
3.1 * (1.7 + 2.1) / 5.0
\end{verbatim}
写成后缀表达式就是
\begin{verbatim}
3.1 1.7 2.1 + * 5.0 /
\end{verbatim}
运行规则是: 准备一个堆栈, 依次读入并压入 \verb|3.1|, \verb|1.7|,
\verb|2.1|, 一旦读到一个运算符\verb|+|, 就把前两个操作数都弹出
(LIFO), 然后计算其结果, 作为新的操作数压回去. 这里就是计算
\begin{verbatim}
1.7 2.1 +
\end{verbatim}
然后把计算结果 \verb|3.8| 压入. 继续, 下一个输入是 \verb|*|, 然后再弹
出最后两个压入数据,
\begin{verbatim}
3.1 3.8 *
\end{verbatim}
结果 \verb|11.78| 压入. 再继续, 读入 \verb|5.0|, 压入. 读入 \verb|\|,
\begin{verbatim}
11.78 5.0 /
\end{verbatim}
结果是 \verb|2.356|. 注意这时, 输入已经结束, 而堆栈也是空的, 所以计算
结束. 整个过程中, 没有任何括号的必要, 甚至没有任何优先级的必要.

这里我们其实已经完成了第三步, 如果我们有办法将中缀表达式转换成后缀表达
式. 这件事似乎很复杂, 因为括号除了优先级规定, 还有是否匹配的问题. 但办
法也是用堆栈.

课本 3.6.3 描述了中缀表达式转后缀表达式的过程, 以及后缀表达式的计算过
程. 但没有给出代码. 我们用 STL 提供的工具, 快速实现一下.

接下去的工作作为作业: 请完成一个四则混合运算器.
\begin{itemize}
\item 输入: 一个由数字, 运算符, 空格, 括号等构成的中缀表达式字符串;
\item 输出: 如果输入是合法的中缀表达式, 给出其运算结果, 否则输出错误.
\end{itemize}
具体要求见学在浙大. 

\subsection{队列应用: 排队的估计和模拟}

这个问题是一门叫排队论的分支学科研究的核心问题. 我们来考虑其最基本的模
型之一. 比如, 现在有一个核酸检测点, 共有 $k$ 个检测台. 每个检测台每个
单位时间检测的人数是 $\mu$, 而每个单位时间到达的人数是 $\lambda$. 现在
问, 作为管理者, 应该如何规划队伍的数量, 使得队伍尽可能有序和高效;而作
为一般群众, 我们的应对策略又该是什么? 

\subsection{简陋条件下的数组实现: 堆栈, 队列和模拟链表}

所谓简陋条件, 指的是不动用 STL 这种复杂工具. 可能是因为编译器, 硬件或
工作环境不允许. 也可能仅仅是因为没这个必要. 而解决方案是一致的, 用数组.

几乎任何语言都有数组. 队列的数组实现是自然的, 但仍然需要两个下标记录,
一个记录当前的队列头, 一个记录当前的队列尾. 入队就是在队尾增加一个新元
素, 而出队只要把头位置往后挪一位就行. 从效率出发, 真实删除出队元素是不
可行的. 而对整个队列做重排或重分配, 只会发生在数组长度耗尽的情况. 所以,
首先在初始化时, 数组不要开得太短, 而循环指标也可以减少数组空间用尽发生
的概率. 也就是当数组下标 $k$ 大于等于当前数组元素总数 $N$ 时,
$$
k = k - N.
$$
与之配合, 出队时, 应该把出队元素赋成一个不常用的特殊数, 比如机器最小数. 

数组的堆栈实现也是类似的, 只需要标记当前栈顶的下标就行了. 入栈就是在栈
顶插入新元素, 栈顶下标加一;出栈就是把栈顶元素读出, 同时栈顶下标减一.
它的实现比队列更加方便.

如果要操作的元素对象比较复杂, 在条件简陋时, 可以用指针数组.

最后说一个极端简陋的情形: 不用指针完成链表操作. 又称为模拟链表. 其实,
链表在计算机内部的实现, 就是依赖于类数组操作的. 计算机内部数据都是按一
定的单位大小, 存放在连续的内存空间的, 每一个单位, 都有连续的, 唯一的内
存地址(下标). 因此你只要模拟这种计算机内部逻辑就可以了.

具体说, 就是用两个数组, 一个是数据元素数组 \verb|data[]|, 一个是下标数
组 \verb|next[]|. 两个数组一一对应, \verb|next[i]| 里存放的是数据元素
\verb|data[i]| 的下一个元素的下标. 如果 \verb|next[i] == -1|, 则表示
\verb|NULL|. 而 \verb|head| 也就是 \verb|data[]| 的一个下标(未必是
\verb|0|).
例如:
\begin{verbatim}
3.1 -> 7.2 -> -1.0 -> 5.5 
\end{verbatim}
可以存储为:
\begin{verbatim}
  data[] = {5.5, 6.7, 7.2, 9.3, -1.0, 255, 3.1, 114, 514}
  next[] = { -1,   0,   4,   0,    0,   0,   2,   0,   0}
\end{verbatim}
注意设 \verb|head = 6|. 我们遍历一遍: \verb|head = 6|, 所以第一个元素
是 \verb|data[6] = 3.1|. 然后下一个元素下标是 \verb|next[6] = 2|, 元素
是 \verb|data[2] = 7.2|. 再下一个元素下标是 \verb|next[2] = 4|, 元素是
\verb|data[4] = -1.0|. 再下一个元素下标是 \verb|next[4] = 0|, 元素是
\verb|data[0] = 5.5|. 再下一个元素下标是 \verb|next[0] = -1|, 已经到了
结尾. 遍历完成. 

请自行考虑如何在此框架下实现数组的基本操作和双链表等. (作业?)

\bibliographystyle{plain}

\bibliography{crazyfish.bib}

\end{document}
